%% Description: 
%%
%%

\section {Installation/Quickstart}
\thispagestyle{plain}

Before installing the PyOPC framework, the following three software
packages have to be installed:

\begin{description}
\item[Python:] The Python programming language can be downloaded from
http://www.python.org. It is available for a variety of operating systems.
{\bf The Python version must be at least 2.4.}

\item[Zolera Soap Infrastructure (ZSI):] The ZSI framework is used for
parsing and generating the SOAP messages. It is available from
http://pywebsvcs.sourceforge.net/. ZSI is still under development, therefore
different releases may not work. The ZSI-2.0\_rc3 release is known to work
with PyOPC.

\item[Twisted:] The Twisted server framework is used for the server
functionality. It is available from http://twistedmatrix.com/. All
recently released versions should be appropriate.
\end{description}

Installation instructions for the above software packages should be
available at the given websites.

PyOPC does currently not have an installer but is nevertheless
relatively easy to install. At first the PyOPC package has to be
decompressed. Then Python has to be informed where to find it. This is
done by adding the location of the PyOPC variable to the environment
variable {\sl PYTHONPATH}.

If everything is installed correctly, the next step may be to test the
installation. PyOPC contains extensive unit tests in the subdirectory ``test''.
These tests can either be run altogether by executing ``runtests.sh'' or 
selectively via the ``trial'' command from the Twisted framework. Hopefully,
all tests will pass\footnote{On some slower machines, certain server operations
may fail as they rely on a predefined execution time of certain operations.}. 

If all goes well, PyOPC is ready to use. As a quickstart, an existing
OPC XML-DA server may be queried directly from python such as shown in
listing \ref{ex_quickstart}\footnote{A demo server, implemented with
PyOPC, is set up at http://violin.qwer.tk:8000/, which may, however,
not be available all the time. There are several other public OPC
XML-DA compliant servers available. Some addresses for such servers
may be found at http://www.opcfoundation.org, moreover Advosol also
offers access to some demo servers (see http://www.advosol.com).}.

\lstset{language=C}
\begin{lstlisting}[caption={Accessing a remote OPC XML-DA server}
                   ,label=ex_quickstart] 
from PyOPC.XDAClient import XDAClient

address='http://path/to/server'
xda = XDAClient(OPCServerAddress=address)
xda.GetStatus()
\end{lstlisting}

