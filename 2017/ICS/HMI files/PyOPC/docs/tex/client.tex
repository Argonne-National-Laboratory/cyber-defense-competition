%% Description: 
%%
%%

\section {Client Functionality}
\thispagestyle{plain}

The PyOPC framework enables access of OPC XML-DA compliant servers by
providing classes that can be used to easily create OPC XML-DA
clients.

PyOPC offers two different classes for this task:

\begin{itemize}
\item The {\sl XDAClient} class that implements simple access of OPC
servers. This class does not offer concurrent connections.

\item PyOPC also provides the more complicated {\sl TWXDAClient}
class, which is based on the Twisted framework. {\sl TWXDAClient}
enables concurrent client connections by utilizing Twisted's event
mechanism.
\end{itemize}

These classes are contained in two different Python modules, namely
{\sl XDAClient} and {\sl TWXDAClient}) that have to be imported before
the client classes can be used. As already shown in Listing
\ref{ex_global}, global options can be defined during object creation,
which then apply to all OPC operations that are handled by this client
object.

Most of these options are OPC-specific and are described in
\cite{OPCXMLDA}.  However, the following options are PyOPC-specific or are 
automatically handled by the client object:

\begin{description}
\item[OPCServerAddress:] This option specifies the address of the OPC
XML-DA server, such as {\sl http://path/to/server}. The
OPCServerAddress option is mandatory and can only be applied during
client object creation.

\item[ClientRequestHandle/ClientItemHandle:] These options may help
the OPC client and server to distinguish between different client
requests. If these options are not specified, they will be
automatically generated by PyOPC.
\end{description}

%% All PyOPC-based twisted clients log the SOAP messages to a file called
%% ``soap.log'' in the current working directory. This file can be
%% utilized by developers to debug certain problems.

\subsection{Building OPC XML-DA Clients with the PyOPC XDAClient class}

Listing \ref{ex_xdaclient} shows example code of a PyOPC
XDAClient-based client\footnote{This sample code can also be found in
the file {\sl samples/clients/simple.py} in the PyOPC distribution}
that first retrieves the server status, browses the root item and
reads an item:


\lstset{language=C}
\begin{lstlisting}[caption={Sample client code based on the PyOPC XDAClient 
module}
                   ,label=ex_xdaclient] 
from PyOPC.OPCContainers import *
from PyOPC.XDAClient import XDAClient

def print_options((ilist,options)):
    print ilist; print options; print
    
address='http://violin.qwer.tk:8000/'

xda = XDAClient(OPCServerAddress=address,
                ReturnErrorText=True)

print_options(xda.GetStatus())
print_options(xda.Browse())
print_options(xda.Read([ItemContainer(ItemName='simple_item', 
                                      MaxAge=500)], 
                       LocaleID='en-us'))
\end{lstlisting}

Line 1 and 2 import the needed PyOPC modules. In Line 4 a simple function
is defined that prints a list of ItemContainer objects ({\sl ilist}) and
the global options Python dictionary ({\sl Options}).

In line 9, the client object is created: As global options, the server
address is specified and {\sl ReturnErrorText = True} denotes that the
client requests verbose error descriptions.

Lines 12, 13 and 14 show the three different OPC operations. The
return parameters of these operations are a list of ItemContainer
objects and the global options (a Python dictionary), which are both
handled by the function {\sl print\_options}.

\subsection{Building OPC XML-DA Clients with the PyOPC TWXDAClient class}

The {\sl XDAClient} module has the disadvantage that operations can
only be handled sequentially. When certain OPC operations take
significantly longer than others, it is a better solution to execute
the requests in parallel. The Twisted framework introduces an
event-based mechanism that is utilized by the the {\sl TWXDAClient},
so that concurrent server requests can be made\footnote{This
event-based mechanism must not be confused with
multi-threading.}. However, this Twisted-based client class is more
complex than its simpler alternative. More information about the
Twisted framework and its underlying concepts can be found in
\cite{TWISTED}.

Listing \ref{ex_twxdaclient} implements the same functionality as 
listing \ref{ex_xdaclient} but executes the three OPC operations 
concurrently:

\lstset{language=C}
\begin{lstlisting}[caption={Sample client code based on the PyOPC TWXDAClient 
module}
                   ,label=ex_twxdaclient] 
from PyOPC.OPCContainers import *
from PyOPC.TWXDAClient import TWXDAClient
from twisted.internet import reactor

OPERATIONS = 3

def print_options((ilist,options)):
    print ilist; print options; print
    global OPERATIONS
    OPERATIONS -= 1
    if OPERATIONS == 0:
        reactor.stop()

def handleError(failure):
    print "An Error occured"
    print failure.getTraceback()
    reactor.stop()
    
address='http://violin.qwer.tk:8000/'

xda = TWXDAClient(OPCServerAddress=address,
                ReturnErrorText=True)

d = xda.twGetStatus()
d.addCallback(print_options)
d.addErrback(handleError)

d = xda.twBrowse()
d.addCallback(print_options)
d.addErrback(handleError)

d = xda.twRead([ItemContainer(ItemName='simple_item', MaxAge=500)],
               LocaleID='en-us')
d.addCallback(print_options)
d.addErrback(handleError)

reactor.run()
\end{lstlisting}

In line 3, certain Twisted modules are imported. Line 24-26 show how
an OPC operation is done in ``Twisted style'': first, the method {\sl
twGetStatus} is called that returns a deferred. Then two functions are
attached to this deferred, namely a ``callback'' method, which prints
the results of the OPC operation and an ``errback'' method, which is
executed when an error (failure) is returned.

In line 37, all deferreds are initialized, therefore the Twisted
reactor is started, which triggers all deferreds. This way, all OPC
operations are started and when the requested data is returned, the
appropriate callback/errback methods are called.

As the execution order of the attached callback methods cannot be
predicted, a global variable {\sl OPERATIONS} is defined, which is
used by the function {\sl print\_options} to stop the Twisted reactor
when all pending server requests have finished.

